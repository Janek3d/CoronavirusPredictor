\documentclass[conference]{IEEEtran}
\IEEEoverridecommandlockouts
% The preceding line is only needed to identify funding in the first footnote. If that is unneeded, please comment it out.
\usepackage{cite}
\usepackage{amsmath,amssymb,amsfonts}
\usepackage{algorithmic}
\usepackage{graphicx}
\usepackage{textcomp}
\usepackage{xcolor}
\def\BibTeX{{\rm B\kern-.05em{\sc i\kern-.025em b}\kern-.08em
    T\kern-.1667em\lower.7ex\hbox{E}\kern-.125emX}}
\begin{document}

\title{Comparison of two selected ML models for predicting deaths due to COVID-19 outbreak}

\author{\IEEEauthorblockN{1\textsuperscript{st} Szymon Krasuski}
\IEEEauthorblockA{\textit{Warsaw University of Technology} \\
Warsaw
}
\and
\IEEEauthorblockN{2\textsuperscript{nd} Janusz Mikłuszka}
\IEEEauthorblockA{\textit{Warsaw University of Technology} \\
Czeladź
}
}


\maketitle

\begin{abstract}
\dots
\end{abstract}

\begin{IEEEkeywords}
COVID-19, Machine learning, Autoregression.
\end{IEEEkeywords}

\section{Introduction}
Current coronavirus pandemic started in chinese Wuhan, in December 2019. On 11 March WHO made the assessment that COVID-19 can be
 characterized as pandemic which became global problem. On 4 March first case in Poland appeared. From this time number of confirmed cases was steadily increasing.
 Month later there was 5000 confirmed cases. 

\dots

\section{Problem Description}
Using databases of confirmed cases, death cases and recovered cases of COVID-19 we can create model which can give us predictions of future cases.
 Based on these prediction decisions can be made regarding public restrictions and other regulations.
% może jakieś wykresy wzrostu dać czy co innego

\dots

\section{Domain Implementation}
% tego nie jestem pewien co powinno być, może coś w czym robimy
We are using Python enviroment to create application which based on provided data and prediction horizon, makes predictions of number of deaths per day.
 We primarily uses autoregressive (AR) models which are primarily used to describe time-varying processes in nature. This type of model specifies the output
  variable based on its previous values and stochastic term. Along with moving-average (MA) model it is component of more general autoregressive-moving-average
  model (ARMA). Dynamic of the autoregressive model of order p AR(p) is given by:
\begin{equation}
    X_t = c + \sum_{i=1}^{p} (\phi_iX_{t-i}) + \epsilon_t
\end{equation}
where
p - order of the model\newline
$ \phi $ - parameters of the model\newline
c - constant\newline
$\epsilon_t$ - white noise\newline


Moving average model of order q MA(q) is written:
\begin{equation}
    X_t = \mu + \epsilon_t + \sum_{i=1}^{q} (\theta_i\epsilon_{t-i})
\end{equation}

ARMA(p,q) notation refers to AR(p) and MA(q) models:
\begin{equation}
    X_t = c + \epsilon_t +\sum_{i=1}^{p} (\phi_iX_{t-i}) + \sum_{i=1}^{q} (\theta_i\epsilon_{t-i})
\end{equation}
\dots

\section{Machine Learning Models}
We are using AutoReg function from statsmodel library. This function uses Autoregressive. It uses only vector of values we want to predict.
 We also need to define order of the model.
\dots

\section{Results}
\dots

\section{Summary}
\dots


\begin{thebibliography}{00}
\bibitem{b1} https://www.sciencedirect.com/science/article/pii/0047259X85900272
\bibitem{b2} https://link.springer.com/chapter/10.1007/978-1-4612-1694-0\_12
\bibitem{b3} https://journals.ametsoc.org/doi/full/10.1175/JHM517.1
\bibitem{b4}
\bibitem{b5}
\bibitem{b6}
\bibitem{b7}
\end{thebibliography}
\vspace{12pt}

\end{document}
